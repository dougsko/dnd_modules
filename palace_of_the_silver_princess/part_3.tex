\documentclass[palace_of_the_silver_princess]{subfiles}

\begin{document}
\fontfamily{ppl}\selectfont
\clearpage

\twocolumn[{\section{Part 3: Key To The Entrance Level}}]
\markright{Part 3: Key To The Entrance Level}

\subsubsection{1}
\begin{quotebox}
    The entrance way seems to be impassable. A massive and
    foreboding double portcullis blocks the entryway of a 30’
    wide corridor. A breeze is gently blowing from the palace
    corridor and it carries with it the dust of decayed stone
    and the smell of decaying bodies. Occasionally sounds of
    pain, fright, and hunger can be heard, but they are far away
    and sometimes muffled, so that all that may be heard is a
    short piercing scream and then total silence.
\end{quotebox}

Due to the width of the corridor and the natural lighting (be it
sunlight or moonlight), vision is clear to the end of the
corridor, at which point two openings, both leading south, and
also blocked by bars, can be seen.

The party cannot see what is beyond the two openings. Sounds
coming from deep in the palace can be heard every few minutes.
Once inside the party will hear, just beyond the double
portcullis, four enchanted voices. One emits a faraway piercing
scream that is soon muffled, the second enchanted voice imitates
someone in pain, the third one screams in fright and the last
one wails in hunger.

There is no monster or treasure in this area.

\subsubsection{1A}
\begin{quotebox}
    Two passageways can be seen here. Each is behind a
    double portcullis. The first one leads south, while
    the second extends west.
\end{quotebox}

It will take a total of 20 strength points to raise
either of these portcullises.

There is no monster or treasure in this area.

\subsubsection{1B}
\begin{quotebox}
    There are two passageways here blocked by a double portcullis. One
    of the passages leads south, the other east. Beyond 15’, down either
    passage, vision is impaired and nothing but blackness can be seen
    (this applies to the other passage as well). The south passage way
    seems to be drier than the east one. The eastern passageway has a
    hint of moisture in the air and dampness can be felt on the wall
    just inside the portcullis.
\end{quotebox}

There is no monster or treasure in this area.

\subsubsection{1C}
\begin{quotebox}
    The walls of this room are collapsing. Moisture clings to everything
    and purple moss grows everywhere throughout the room. Torches
    flicker and sputter as if they are not getting enough oxygen to
    burn. The air feels heavy and hard to breathe. A sweet smell fills
    the room and gets stronger as time passes.
\end{quotebox}

The \textbf{purple moss} is a type of plant that thrives on moisture and
flesh.  The sweet smell the party has detected is a sleeping gas
produced by the plant. Once the victim is asleep the moss will quickly
cover the body and devour it in less than an hour. It then hides the
bones of its dinner by covering them and soon they become
indistinguishable from any other normal mound of moss. Each player will
have to make a successful \textbf{DC 10 Constitution saving throw} in
order to avoid being affected by the sleep gas.  The purple moss cannot
be harmed except by normal or magical fire.

There is no treasure in this area.

% You can optionally not include the background by saying
% begin{monsterboxnobg}
\begin{monsterbox}{Purple Moss}
    \textit{Medium plant, unaligned}\\
    \hline
    \basics[%
        armorclass = 5,
        hitpoints  = 18 (4d8),
        speed      = 5 ft
    ]
    \hline
    \stats[
        STR = \stat{3},
        DEX = \stat{1},
        CON = \stat{10},
        INT = \stat{1},
        WIS = \stat{3},
        CHA = \stat{1}
    ]
    \hline
    \details[
        languages = {---},
        challenge = {1/4},
        damagevulnerabilities = {fire},
    ]
    \hline
    \\[1mm]
    \begin{monsteraction}[Sleeping gas]
        A creature within 30 ft. of the moss must make a successful DC
        10 Constitution saving throw or fall asleep.
    \end{monsteraction}
    % \monstersection{Actions}
    % \begin{monsteraction}[Generate text]
    % This one can generate tremendous amounts of text! 
    % \end{monsteraction}
\end{monsterbox}

\subsubsection{1D}
\begin{quotebox}
    This huge cave area is filled with the sweet smell of fresh water.
    The source is obviously a rather large grey stone pool of water that
    almost covers the entire floor of the cavern. Occasionally bubbles
    rise to the surface of the water, but apart from that the water is
    quiet. A small ledge circles one end of the pool. This ledge is wide
    enough for one fully armored person to inch around the pool to the
    other side where an opening can be seen.
\end{quotebox}

If the party disturbs the water, 12 \textbf{bubbles} will rise to the
surface to defend their lair. The bubbles will attempt to surprise the
party by rising to the surface all at once.  The pool is 15’ deep in its
deepest point, and 4’ deep at its shallowest point.  If the victim
cannot be saved, the bubble will expel the dead victim and rise to the
surface to attack again. The body, unless armored, will float to the
surface.

\begin{monsterbox}{Bubble}
    \textit{Medium construct, unaligned}\\
    \hline
    \basics[%
        armorclass = 9,
        hitpoints  = 2,
        speed      = 30 ft.
    ]
    \hline
    \stats[
        STR = \stat{1},
        DEX = \stat{5},
        CON = \stat{10},
        INT = \stat{1},
        WIS = \stat{1},
        CHA = \stat{1}
    ]
    \hline
    \details[
        languages = {---},
        challenge = {1/4},
    ]
    \hline
    \\[1mm]
    %\begin{monsteraction}[Sleeping gas]
    %    A creature within 30 ft. of the moss must make a successful DC
    %    10 Constitution saving throw or fall asleep.
    %\end{monsteraction}
     \monstersection{Actions}
     \begin{monsteraction}[Paralyze]
         The target must succeed on a DC 13 Constitution saving throw or
         be paralyzed.  If a bubble manages to successfully paralyze
         someone, it will engulf that victim and then sink back down to
         the bottom of the pool.

         The victim will suffocate in 2-5 (1d4 + 1) rounds unless someone
         manages to kill the enclosing bubble.
     \end{monsteraction}
\end{monsterbox}

If the party manages to successfully kill all the bubbles, their
treasure may be found at the deepest point of the pool. \textbf{A small
bag of 133 gold pieces and one silver wolf-head ring (value: 33 gold
pieces)} will be found if the pool is searched.

Stairs lead down the passageway from the pool to a dead end. This area
may be opened up by the DM.

\begin{quotebox}
    This small rectangular cave opens up at the base of long steep
    stairs. Red coarse sand surrounds a small grey pool of water. The
    ledge around the water is wide enough for one fully armored person
    to walk with ease.
\end{quotebox}

The sand is colored red, and if the party rinses the sand they will
discover that it is normal coarse sand but once dry becomes red again.
The water does not contain any monsters, but if the party examines the
pool carefully, they will find that it is spring fed. The drain appears
to be near the southern end of the pool. If this is plugged and the pool
is allowed to flood, the adventurers will discover that the cave floor
gently slopes to the south. After several hours, a steady stream will
appear. After several days, the entire basin at the base of the southern
stairs will be completely flooded. (If the party does block the drain,
note it for future reference.)

\subsubsection{1F}

This is an empty room. The DM may wish to insert an encounter of his
or her own choosing here or stock the room with valueless items designed
to waste a party’s time. This also applies to the other empty rooms
provided throughout this module.

\subsubsection{2}

\begin{quotebox}
    Reed pens, dried ink wells. and hundreds of scraps of paper litter
    this large room. There are several huge oak tables overturned near
    the southeast corner. This room appears to have been some kind of
    study, classroom or library. There are no books or intact scrolls
    anywhere to be seen.
\end{quotebox}

Hidden behind the tables is a family of five kobolds.  If the party
decides to search the room, or they discover the kobolds, the kobolds
will fight. Otherwise, they will remain hidden until the danger passes.
Buried in the rubble of the kobolds’ nest are \textbf{50 copper pieces}.

\begin{monsterbox}{Kobold}
    \textit{Small humanoid (kobold), lawful evil}\\
    \hline
    \basics[%
        armorclass = 12,
        hitpoints  = 5 (2d6 - 2),
        speed      = 30 ft.
    ]
    \hline
    \stats[
        STR = \stat{7},
        DEX = \stat{15},
        CON = \stat{9},
        INT = \stat{8},
        WIS = \stat{7},
        CHA = \stat{8}
    ]
    \hline
    \details[
        senses = {darkvision 60 ft., passive Perception 8},
        languages = {Common, Draconic},
        challenge = {1/8 (25 XP)},
    ]
    \hline
    \\[1mm]
    \begin{monsteraction}[Sunlight Sensitivity]
        While in sunlight, the kobold has disadvantage on attack rolls,
        as well as on Wisdom (Perception) checks that rely on sight.
    \end{monsteraction}

    \begin{monsteraction}[Pack Tactics]
        The kobold has advantage on an attack roll against a creature
        if at least one ofthe kobold's allies is within 5 feet of the
        creature and the ally isn't incapacitated.
    \end{monsteraction}
     \monstersection{Actions}
     \begin{monsteraction}[Dagger]
         \textit{Melee Weapon Attack:} +4 to hit, reach 5 ft., one target.
         \textit{Hit:} 4 (1d4 + 2) piercing damage.
     \end{monsteraction}

     \begin{monsteraction}[Sling]
         \textit{Ranged Weapon Attack:} +4 to hit, range 30/120 ft.,
         one target. 
         \textit{Hit:} 4 (1d4 + 2) bludgeoning damage.
     \end{monsteraction}
\end{monsterbox}

\subsubsection{3}

\begin{quotebox}
    Rotten bags of grain, old brooms, and three decaying beer barrels
    full of vinegar are all that remain in this shelved room. It appears
    to once have been a store room. It is not obvious as to whether the
    inhabitants left the grain and beer because they could not transport
    them or because they had no choice but to leave them.
\end{quotebox}

If the players examine the barrels they will discover that one is full
of pickled snakes. If they touch the sacks of grain, the material, due
to its age, will come off in their hands in small patches. The grain
itself has a horrible smell, as does the vinegar in the barrels.

There is no monster or treasure in the room.

\subsubsection{4}

\begin{quotebox}
    This area was a kitchen. There are many wooden trenchers, spoons,
    and knives scattered about the tables and floors. Three large tubs
    full of water sit on stools near the fireplace. One is full of green
    fungus. A pile of grease soaked rags lies in one corner of the room
    near a keg of dried beans. Pots and other assorted dishes and
    cooking utensils are also lying strewn about the room and are beyond
    cleaning or repair.
\end{quotebox}

Hidden in the rags is a spitting cobra.  It will only attack if
disturbed, otherwise it will remain quiet as it is sleeping.

The green fungus will leave a horrible, sickening, skunk-like smell on
whatever comes in contact with it. The smell will linger for 3-18 days.
\textbf{A small fungus encrusted gold ring} is at the bottom of the
fungus. If the ring is cleaned, players will discover the initials A. E.
S. carved into it.



\end{document}
