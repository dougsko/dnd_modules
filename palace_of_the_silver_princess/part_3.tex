\documentclass[palace_of_the_silver_princess]{subfiles}

\begin{document}
\fontfamily{ppl}\selectfont
\clearpage

\twocolumn[{\section{Part 3: Key To The Entrance Level}}]
\markright{Part 3: Key To The Entrance Level}

\subsubsection{1}
\begin{quotebox}
    The entrance way seems to be impassable. A massive and
    foreboding double portcullis blocks the entryway of a 30’
    wide corridor. A breeze is gently blowing from the palace
    corridor and it carries with it the dust of decayed stone
    and the smell of decaying bodies. Occasionally sounds of
    pain, fright, and hunger can be heard, but they are far away
    and sometimes muffled, so that all that may be heard is a
    short piercing scream and then total silence.
\end{quotebox}

Due to the width of the corridor and the natural lighting (be it
sunlight or moonlight), vision is clear to the end of the
corridor, at which point two openings, both leading south, and
also blocked by bars, can be seen.

The party cannot see what is beyond the two openings. Sounds
coming from deep in the palace can be heard every few minutes.
Once inside the party will hear, just beyond the double
portcullis, four enchanted voices. One emits a faraway piercing
scream that is soon muffled, the second enchanted voice imitates
someone in pain, the third one screams in fright and the last
one wails in hunger.

There is no monster or treasure in this area.

\subsubsection{1A}
\begin{quotebox}
    Two passageways can be seen here. Each is behind a
    double portcullis. The first one leads south, while
    the second extends west.
\end{quotebox}

It will take a total of 20 strength points to raise
either of these portcullises.

There is no monster or treasure in this area.

\subsubsection{1B}
\begin{quotebox}
    There are two passageways here blocked by a double portcullis. One
    of the passages leads south, the other east. Beyond 15’, down either
    passage, vision is impaired and nothing but blackness can be seen
    (this applies to the other passage as well). The south passage way
    seems to be drier than the east one. The eastern passageway has a
    hint of moisture in the air and dampness can be felt on the wall
    just inside the portcullis.
\end{quotebox}

There is no monster or treasure in this area.

\subsubsection{1C}
\begin{quotebox}
    The walls of this room are collapsing. Moisture clings to everything
    and purple moss grows everywhere throughout the room. Torches
    flicker and sputter as if they are not getting enough oxygen to
    burn. The air feels heavy and hard to breathe. A sweet smell fills
    the room and gets stronger as time passes.
\end{quotebox}

The \textbf{purple moss} is a type of plant that thrives on moisture and
flesh.  The sweet smell the party has detected is a sleeping gas
produced by the plant. Once the victim is asleep the moss will quickly
cover the body and devour it in less than an hour. It then hides the
bones of its dinner by covering them and soon they become
indistinguishable from any other normal mound of moss. Each player will
have to make a successful \textbf{DC 10 Constitution saving throw} in
order to avoid being affected by the sleep gas.  The purple moss cannot
be harmed except by normal or magical fire.

There is no treasure in this area.

% You can optionally not include the background by saying
% begin{monsterboxnobg}
\begin{monsterbox}{Purple Moss}
    \index[monsters]{Purple Moss}
    \textit{Medium plant, unaligned}\\
    \hline
    \basics[%
        armorclass = 5,
        hitpoints  = 18 (4d8),
        speed      = 5 ft
    ]
    \hline
    \stats[
        STR = \stat{3},
        DEX = \stat{1},
        CON = \stat{10},
        INT = \stat{1},
        WIS = \stat{3},
        CHA = \stat{1}
    ]
    \hline
    \details[
        languages = {---},
        challenge = {1/4},
        damagevulnerabilities = {fire},
    ]
    \hline
    \\[1mm]
    \begin{monsteraction}[Sleeping gas]
        A creature within 30 ft. of the moss must make a successful DC
        10 Constitution saving throw or fall asleep.
    \end{monsteraction}
    % \monstersection{Actions}
    % \begin{monsteraction}[Generate text]
    % This one can generate tremendous amounts of text! 
    % \end{monsteraction}
\end{monsterbox}

\subsubsection{1D}
\begin{quotebox}
    This huge cave area is filled with the sweet smell of fresh water.
    The source is obviously a rather large grey stone pool of water that
    almost covers the entire floor of the cavern. Occasionally bubbles
    rise to the surface of the water, but apart from that the water is
    quiet. A small ledge circles one end of the pool. This ledge is wide
    enough for one fully armored person to inch around the pool to the
    other side where an opening can be seen.
\end{quotebox}

If the party disturbs the water, 12 \textbf{bubbles} will rise to the
surface to defend their lair. The bubbles will attempt to surprise the
party by rising to the surface all at once.  The pool is 15’ deep in its
deepest point, and 4’ deep at its shallowest point.  If the victim
cannot be saved, the bubble will expel the dead victim and rise to the
surface to attack again. The body, unless armored, will float to the
surface.

\begin{monsterbox}{Bubble}
    \index[monsters]{Bubble}
    \textit{Medium construct, unaligned}\\
    \hline
    \basics[%
        armorclass = 9,
        hitpoints  = 2,
        speed      = 30 ft.
    ]
    \hline
    \stats[
        STR = \stat{1},
        DEX = \stat{5},
        CON = \stat{10},
        INT = \stat{1},
        WIS = \stat{1},
        CHA = \stat{1}
    ]
    \hline
    \details[
        languages = {---},
        challenge = {1/4},
    ]
    \hline
    \\[1mm]
    %\begin{monsteraction}[Sleeping gas]
    %    A creature within 30 ft. of the moss must make a successful DC
    %    10 Constitution saving throw or fall asleep.
    %\end{monsteraction}
     \monstersection{Actions}
     \begin{monsteraction}[Paralyze]
         The target must succeed on a DC 13 Constitution saving throw or
         be paralyzed.  If a bubble manages to successfully paralyze
         someone, it will engulf that victim and then sink back down to
         the bottom of the pool.

         The victim will suffocate in 2-5 (1d4 + 1) rounds unless someone
         manages to kill the enclosing bubble.
     \end{monsteraction}
\end{monsterbox}

If the party manages to successfully kill all the bubbles, their
treasure may be found at the deepest point of the pool. \textbf{A small
bag of 133 gold pieces and one silver wolf-head ring (value: 33 gold
pieces)} will be found if the pool is searched.

Stairs lead down the passageway from the pool to a dead end. This area
may be opened up by the DM.

\begin{quotebox}
    This small rectangular cave opens up at the base of long steep
    stairs. Red coarse sand surrounds a small grey pool of water. The
    ledge around the water is wide enough for one fully armored person
    to walk with ease.
\end{quotebox}

The sand is colored red, and if the party rinses the sand they will
discover that it is normal coarse sand but once dry becomes red again.
The water does not contain any monsters, but if the party examines the
pool carefully, they will find that it is spring fed. The drain appears
to be near the southern end of the pool. If this is plugged and the pool
is allowed to flood, the adventurers will discover that the cave floor
gently slopes to the south. After several hours, a steady stream will
appear. After several days, the entire basin at the base of the southern
stairs will be completely flooded. (If the party does block the drain,
note it for future reference.)

\subsubsection{1F}

This is an empty room. The DM may wish to insert an encounter of his
or her own choosing here or stock the room with valueless items designed
to waste a party’s time. This also applies to the other empty rooms
provided throughout this module.

\subsubsection{2}

\begin{quotebox}
    Reed pens, dried ink wells. and hundreds of scraps of paper litter
    this large room. There are several huge oak tables overturned near
    the southeast corner. This room appears to have been some kind of
    study, classroom or library. There are no books or intact scrolls
    anywhere to be seen.
\end{quotebox}

Hidden behind the tables is a family of five kobolds.  If the party
decides to search the room, or they discover the kobolds, the kobolds
will fight. Otherwise, they will remain hidden until the danger passes.
Buried in the rubble of the kobolds’ nest are \textbf{50 copper pieces}.

\begin{monsterbox}{Kobold}
    \index[monsters]{Kobold}
    \textit{Small humanoid (kobold), lawful evil}\\
    \hline
    \basics[%
        armorclass = 12,
        hitpoints  = 5 (2d6 - 2),
        speed      = 30 ft.
    ]
    \hline
    \stats[
        STR = \stat{7},
        DEX = \stat{15},
        CON = \stat{9},
        INT = \stat{8},
        WIS = \stat{7},
        CHA = \stat{8}
    ]
    \hline
    \details[
        senses = {darkvision 60 ft., passive Perception 8},
        languages = {Common, Draconic},
        challenge = {1/8 (25 XP)},
    ]
    \hline
    \\[1mm]
    \begin{monsteraction}[Sunlight Sensitivity]
        While in sunlight, the kobold has disadvantage on attack rolls,
        as well as on Wisdom (Perception) checks that rely on sight.
    \end{monsteraction}

    \begin{monsteraction}[Pack Tactics]
        The kobold has advantage on an attack roll against a creature
        if at least one of the kobold's allies is within 5 feet of the
        creature and the ally isn't incapacitated.
    \end{monsteraction}
     \monstersection{Actions}
     \begin{monsteraction}[Dagger]
         \textit{Melee Weapon Attack:} +4 to hit, reach 5 ft., one target.

         \textit{Hit:} 4 (1d4 + 2) piercing damage.
     \end{monsteraction}

     \begin{monsteraction}[Sling]
         \textit{Ranged Weapon Attack:} +4 to hit, range 30/120 ft.,
         one target. 

         \textit{Hit:} 4 (1d4 + 2) bludgeoning damage.
     \end{monsteraction}
\end{monsterbox}

\subsubsection{3}

\begin{quotebox}
    Rotten bags of grain, old brooms, and three decaying beer barrels
    full of vinegar are all that remain in this shelved room. It appears
    to once have been a store room. It is not obvious as to whether the
    inhabitants left the grain and beer because they could not transport
    them or because they had no choice but to leave them.
\end{quotebox}

If the players examine the barrels they will discover that one is full
of pickled snakes. If they touch the sacks of grain, the material, due
to its age, will come off in their hands in small patches. The grain
itself has a horrible smell, as does the vinegar in the barrels.

There is no monster or treasure in the room.

\subsubsection{4}

\begin{quotebox}
    This area was a kitchen. There are many wooden trenchers, spoons,
    and knives scattered about the tables and floors. Three large tubs
    full of water sit on stools near the fireplace. One is full of green
    fungus. A pile of grease soaked rags lies in one corner of the room
    near a keg of dried beans. Pots and other assorted dishes and
    cooking utensils are also lying strewn about the room and are beyond
    cleaning or repair.
\end{quotebox}

Hidden in the rags is a \textbf{poisonous snake}.  It will only attack
if disturbed, otherwise it will remain quiet as it is sleeping.

The green fungus will leave a horrible, sickening, skunk-like smell on
whatever comes in contact with it. The smell will linger for 3-18 days.
\textbf{A small fungus encrusted gold ring} is at the bottom of the
fungus. If the ring is cleaned, players will discover the initials A. E.
S. carved into it.

\begin{monsterbox}{Poisonous Snake}
    \index[monsters]{Poisonous Snake}
    \textit{Tiny beast, unaligned}\\
    \hline
    \basics[%
        armorclass = 13,
        hitpoints  = 2 (1d4),
        speed      = {30 ft., swim 30 ft.}
    ]
    \hline
    \stats[
        STR = \stat{2},
        DEX = \stat{16},
        CON = \stat{11},
        INT = \stat{1},
        WIS = \stat{10},
        CHA = \stat{3}
    ]
    \hline
    \details[
        senses = {blindsight 10 ft., passive Perception 10},
        languages = {---},
        challenge = {1/8 (25 XP)},
    ]
    \hline
    \\[1mm]
     \monstersection{Actions}
     \begin{monsteraction}[Bite]
         \textit{Melee Weapon Attack:} +5 to hit , reach 5 ft., one
         target.

         \textit{Hit:} 1 piercing damage, and the target must make a DC
         10 Constitution saving throw, taking 5 (2d4) poison damage on a
         failed save, or half as much damage on a successful one.
     \end{monsteraction}
\end{monsterbox}

\subsubsection{5}

\begin{quotebox}
    At first it is hard to determine what this room was used for, but
    after careful observation it becomes apparent that it once was a
    dining hall, but now is a complete wreck. Tables, benches and stools
    have been smashed into hundreds of pieces, torch sconces have been
    ripped out of the walls, graffiti covers one wall and garbage is
    piled about the room in small, stinking heaps. The remains of
    several fires can be seen near the center of the room.
\end{quotebox}

Lying in wait under a table top is a \textbf{carrion crawler}.  It will
wait until someone gets close enough for it to grab. It is not looking
for a fight, as it is recovering from battle wounds recently sustained,
but it will not flee either. (The carrion crawler was wounded by the
dead soldiers that will be found in room \textbf{EL 7}).

If the players examine the fire remains carefully there is a 25 \%
chance per examiner that they will be able to discern from discarded
tinder boxes and other tools of orcish make that the fires seem to have
been set by orcs. In addition, the graffiti scrawled on the walls is
full of cruel orcish boasts and threats recognizable to any character
who can speak and read orcish.

One of the bits of wood lying on the floor is actually a \textbf{Wand of
Secrets}, however the players will not know that until they have
identified it.  Each player has a 10\% chance, per round of searching,
to find the wand. In the process, a ring of what appears to be jailers’
keys will be found. There are 6 keys on the ring, each exactly alike.
These keys have no cash value but will open the cells located at
\textbf{EL 32}.

\subtitlesection{Wand of Secrets}
{Wand, uncommon}
\index[magic]{Wand of Secrets}

The wand has 3 charges. While holding it. You can use an action to
expend 1 of its charges, and if a secret door or trap is within 30 feet
of you, the wand pulses and points at the one nearest to you. The wand
regains 1d3 expended charges daily at dawn.

\begin{monsterbox}{Carrion Crawler}
    \index[monsters]{Carrion Crawler}
    \textit{Large monstrosity, unaligned}\\
    \hline
    \basics[%
        armorclass = 13,
        hitpoints  = 25,
        speed      = {30 ft., climb 30 ft.}
    ]
    \hline
    \stats[
        STR = \stat{14},
        DEX = \stat{13},
        CON = \stat{16},
        INT = \stat{1},
        WIS = \stat{12},
        CHA = \stat{5}
    ]
    \hline
    \details[
        senses = {darkvision 60 ft., passive Perception 13},
        languages = {---},
        challenge = {2 (450 XP)},
    ]
    \hline
    \\[1mm]
    \begin{monsteraction}[Keen Smell]
        The carrion crawler has advantage on Wisdom (Perception) checks
        that rely on smell.
    \end{monsteraction}
    \begin{monsteraction}[Spider Climb]
        The carrion crawler can climb difficult surfaces, including
        upside down on ceilings, without needing to make an ability
        check.
    \end{monsteraction}
    \monstersection{Actions}
    \begin{monsteraction}[Multiattack]
        The carrion crawler makes two attacks: one with its tentacles
        and one with its bite.
    \end{monsteraction}

    \begin{monsteraction}[Tentacles]
        \textit{Melee Weapon Attack:} +8 to hit, reach 10ft., one
        creature. 

        \textit{Hit:} 4 (1d4 + 2) poison damage, and the target must
        succeed on a DC 13 Constitution saving throw or be poisoned for
        1 minute. Until this poison ends, the target is paralyzed. The
        target can repeat the saving throw at the end of each of its
        turns, ending the poison on itself on a success.
    \end{monsteraction}

    \begin{monsteraction}[Bite]
        \textit{Melee Weapon Attack:} +4 to hit, reach 5 ft., one
        target.

        \textit{Hit:} 7 (2d4 + 2) piercing damage.
    \end{monsteraction}
\end{monsterbox}

\subsubsection{6}
\begin{quotebox}
    Many dusty, musty, smelly bedrolls provide the furniture for this
    room that was once a barracks. Six 3’ footlockers are leaning
    sideways against the west wall and are covered in several inches of
    dust. Outlines of weapons and shields can be seen on the wall
    indicating that at one time the walls sported the occupant’ s tools
    of the trade as decorations for the otherwise barren room. The
    room is very large.
\end{quotebox}

If the party decides to search this room, roll for a wandering monster
only once using the \textbf{Wandering Monster Table}. No other monster
may be found while in this area. During the search there is a 20\%
chance per party member searching that \textbf{three strange gold
coin-like octagons} will be found. These octagons can be used to open a
secret compartment in the base of a statue in area \textbf{EL 14}. If
the octagons are sold, their value will be between 10 and 100 gold
pieces each.

\subsubsection{7}
\begin{quotebox}
    This room contains the remains of bunks, bedrolls, round oaken
    tables, stools. benches and dead soldiers which have been beheaded.
    Along the north wall is a line of 6 heads.
\end{quotebox}

There are no intact weapons left in the room, and all the bodies have
apparently been searched thoroughly, leaving nothing of value on them.
Upon closer examination, the players will notice the insignia on the
uniforms of the soldiers. It resembles a wolf's head with an battlement
and ball between the ears, two slanted eyes, an arrow where the nose
should be and a lightning bolt on the arrow. As the party searches the
room, roll for a wandering monsters. If on the first roll none was
indicated roll again. On the last roll if one was indicated the
wandering monster will be two female thieves: \textbf{Candella and
Dutchess}.  Both women will have an above average appearance and will
attempt to use it to their benefit. They will pretend to be young
inexperienced fighters in search of adventure, fame and fortune, but
mostly fortune.  Candella is the spokesman of the two women.

These two thieves will be friendly towards the party, not acting hostile
if they win the initiative. They will politely ask to join the party,
saying that they are not quite as tough or prepared for adventuring as
they had originally thought themselves to be. Duchess will stress her
desire to accompany them, saying she fears that she and her companion
have made a grave error in attempting to venture into the palace ruins
by themselves, especially after seeing the strange 3 headed monsters
they have managed to flee from so far.

Both thieves will have the following on them including normal dungeon
supplies, weapons and thieves tools:

\begin{itemize}
    \item 15 gp
    \item 7 sp
    \item 21 cp
    \item Wolfsbane (Duchess)
    \item Poisoned daggers (poison effective for one attack)
    \item Stand of pearls (value 600 gp)
\end{itemize}

These two thieves may be used by the DM as NPCs (non-player characters)
or as a normal dungeon encounter.

\subsubsection{8}
\begin{quotebox}
    Wind whistles softly through this dark damp cave carrying with it a
    musky smell. In the entrance way of the cave can be seen two sets of
    animal chains. Straw is scattered about the floor, along with jagged
    bones.
\end{quotebox}

If the party opts to enter the cave, they will soon find themselves face
to face with a very hungry and very young \textbf{cave bear cub}.  It
appears to have been abandoned by its mother though there is a 1\%
chance per turn she will return. If the players offer it food (meat) it
will eat it gladly, but warily watch and growl at the players while it
devours the food.

\begin{paperbox}{Capturing The Cub}
If the party captures the cub they will be forced to sell it as they
will find that it is too big, too wild, and too hungry for them to
afford to keep. Its value on the open market is between 200 and 400 gold
pieces. However, the DM may wish to have the cub auctioned off in a
bazaar, or can allow the players to have it tamed and trained at a great
cost. Training can be done only by a skilled animal trainer and will
cost from 200-700 gp and take from 4-24 weeks. This will allow the DM to
continue the game into the city.
\end{paperbox}

\includegraphics[width=\columnwidth]{img/cave_bear.png}

\begin{monsterbox}{Cave Bear Cub}
    \index[monsters]{Cave Bear Cub}
    \textit{Medium beast, unaligned}\\
    \hline
    \basics[%
        armorclass = 11,
        hitpoints  = 15,
        speed      = {40 ft., swim 30 ft.}
    ]
    \hline
    \stats[
        STR = \stat{10},
        DEX = \stat{5},
        CON = \stat{8},
        INT = \stat{1},
        WIS = \stat{6},
        CHA = \stat{3}
    ]
    \hline
    \details[
        skills = {Perception +3},
        senses = {darkvision 60 ft., passive Perception 13},
        languages = {---},
        challenge = {1 (200 XP)},
    ]
    \hline
    \\[1mm]
    \begin{monsteraction}[Keen Smell]
        The bear has advantage on Wisdom (Perception) checks
        that rely on smell.
    \end{monsteraction}
    \monstersection{Actions}
    \begin{monsteraction}[Multiattack]
        The bear makes two attacks: one with its bite
        and one with its claws.
    \end{monsteraction}

    \begin{monsteraction}[Bite]
        \textit{Melee Weapon Attack:} +4 to hit, reach 5 ft., one
        target.

        \textit{Hit:} 5 (1d8) piercing damage.
    \end{monsteraction}
    
    \begin{monsteraction}[Claws]
        \textit{Melee Weapon Attack:} +4 to hit, reach 5 ft., one
        target.

        \textit{Hit:} 7 (1d6 + 4) slashing damage.
    \end{monsteraction}
\end{monsterbox}

\subsubsection{9}
\begin{quotebox}
    This elongated hexagonal room is littered with smelly, moldy, red
    towels. There is also a lot of dried up soft pink soap in broken
    blue ceramic containers, decorated with romantic scenes of mermaids
    swimming about proud ships and singing songs to the sailors. The
    beautiful marble floors are white, veined in black and gold. Each of
    the 6 walls is decorated with ornately carved wooden towel racks and
    copper torch scones which are now tarnished due to lack of care. A
    lovely bench of black marble with white and gold streaks occupies
    the center of the room. A faded red cushion, now ruined by dry rot,
    lies beside the bench.
\end{quotebox}

Hidden in a towel under the bench is a \textbf{gold colored key on a
thin golden chain}. There is only a 15\% chance that the players will
find the key unless they specifically state that they are looking under
the bench, at which point they will discover the key. This key will open
the secret door in room \textbf{EL 12}. If it is sold, the key and chain
together will only bring 1 gold and 6 silver pieces.

\subsubsection{10}
\begin{quotebox}
    In this room, which is shaped exactly like the last one, is a large
    pool. It appears to be filled with clear water. The walls of this
    room are lavishly decorated with murals of water nymphs, ponds with
    long reeds extending upwards to the sun, and brave hunters stalking
    water birds. Here, as in the last room, are more moldy rotten
    towels. There are also seven delicately carved vials of scented bath
    oils, and a rather large peacock feather fan, now rotted, which is
    propped up in one corner.
\end{quotebox}

If the party examines the pool closely, they will discover what appears
to be a rather large diamond embedded in the center of the pool. The gem
is actually the eye of the diger an amoebic monster that seeks rock or
stone areas in which to camouflage itself as a pool. It is incapable
of attacking anyone or anything unless the victim enters the diger’s
‘pool’.

The vials of oils are worth a gold piece each, and the feather fan, due
to its condition, only 5 copper pieces.

Note the false door and secret door.

\includegraphics[width=\columnwidth]{img/diger.png}

\begin{monsterbox}{Diger}
    \index[monsters]{Diger}
    \textit{Medium ooze, unaligned}\\
    \hline
    \basics[%
        armorclass = 8,
        hitpoints  = 33 (6d6 + 12),
        speed      = {10 ft., climb 10 ft., swim 10 ft.}
    ]
    \hline
    \stats[
        STR = \stat{12},
        DEX = \stat{6},
        CON = \stat{14},
        INT = \stat{1},
        WIS = \stat{6},
        CHA = \stat{2}
    ]
    \hline
    \details[
        damageresistances = {acid},
        damageimmunities = {poison},
        skills = {Athletics +3},
        senses = {blindsight 60 ft., passive Perception 8},
        languages = {---},
        challenge = {1 (200 XP)},
    ]
    \hline
    \\[1mm]
    \begin{monsteraction}[Amorphous]
        The diger can move through a space as narrow as 1 inch wide
        without squeezing.
    \end{monsteraction}
    \monstersection{Actions}
    \begin{monsteraction}[Pseudopod]
        \textit{Melee Weapon Attack:} +4 to hit, reach 5 ft., one
        target.

        \textit{Hit:} 4 (1d6 + 1) bludgeoning damage plus 2 (1d4) acid
        damage and the target is grappled (escape DC 10). Until the
        grapple ends, the target is restrained and has disadvantage on
        Strength checks and Strength saving throws, and the target takes
        2 (1d4) acid damage at the start of each of its turn, while it
        is grappled.
    \end{monsteraction}
\end{monsterbox}

\subsubsection{11}
\begin{quotebox}
    Upon entering this room, the first thing noticed is a small, pink
    marble pedestal about dwarf size in height. Any light entering the
    room will gleam off of a small object atop the pedestal. The object
    is silver in color. Other than the pedestal the room seems to be
    empty.
\end{quotebox}

When a character gets within one foot of the pedestal, the silver
pendant on top of the pedestal will begin to radiate a silver glow that
will illuminate the entire room. After one round, hysterical laughter
will seem to come from the pendant, and anyone within a 10’ radius of it
must make a DC 15 Constitution saving throw or fall into a fit of
uncontrollable laughter that will last 3 rounds or until the pendant is
removed from the pedestal. Any character attempting to remove the
pendant must also make a DC 15 Constitution saving throw or else be
likewise stricken. The second character is allowed a + 2 on his or her
save, as is anyone else who tries. However, the pendant can only affect
three people at any given time. All others will be immune until there
are no longer three people in its area of effect. Once the pendant has
been successfully removed, the stricken character will no longer be
affected by the pendant, and all laughter will cease. Characters who
were affected by the pendant will lose 2 points of strength and 1 point
from their constitution for 2-8 turns. The pendant has no sale value.

\subsubsection{12}
\begin{quotebox}
    This hexagonal room, much like the other ones, is decorated with
    mosaic tiles. The mosaic covers the entire room, the walls, the
    floor and ceiling. The scenes are of a red dragon mounted by a man
    in silver and blue armor giving chase to a young maiden wearing a
    silver gown and a silver and ruby coronet. Another scene depicts
    elves playing in the woods while a red dragon watches them from his
    hiding place behind two tall pines. On one wall is a pool of bright
    blue water with a shimmering diamond floating on a lily paid, and
    several mermaids swimming and splashing each other near it. The
    design on the floor shows the maiden, man and dragon curled up
    asleep around a key hole.
\end{quotebox}

Once the party has entered the room, if they examine the murals, the
keyhole in the floor will emit a blue white glow and will last until a
key is placed into it. If the players use one of the jailer's keys
(providing they found them) or any key other than the gold one from
\textbf{EL 9}, a 5’x5’x1’ stone slab will fall from the ceiling over the
spot where the keyhole is located. Characters within that area must make
a DC 15 Dexterity saving throw to avoid being hit by the stone. Any
character caught by the stone will suffer 2-12 points of damage. If the
golden colored key is placed in the keyhole, another keyhole will appear
on the east wall. The second keyhole is opened by the golden key also.
Once placed in the lock and turned, the wall, keyhole, and key will
vanish. A long silver sword — glowing with a bright blue-white light,
suspended in mid air — will appear in their place. If a character
reaches out to touch the sword, a fully armored man (the one depicted in
the murals), will appear beside it, take the sword and attack the person
who was attempting to take the sword. The man is an illusion and will
disappear after 4 rounds. However, characters hit by the illusion will
believe that they have actually sustained damage and will feel “hurt,”
though no damage was actually taken. The illusion is considered to be AC
2. Once the illusion has disappeared, the sword will drop to the floor,
still glowing as it was when the characters fast saw it. All characters
will immediately realize that they took no damage, and characters who
may have been “killed” will discover that they are actually alive and
were only asleep.

If all the party members are “killed”, they will wake up a short time
later. The illusion will be gone and the glowing sword will be lying on
the floor. The illusion will not reappear if they take the sword before
leaving the room.

If the characters decide to touch the sword again, nothing will happen
to them and the sword will “feel good” in their hands. The sword will
always glow when not sheathed. There is no sheath for it in the room,
nor will it fit into a sheath not specifically designed for it. The
magic properties of the sword are as follows:

\subtitlesection{+1 Glowing Longsword}
{Weapon (any), uncommon}
\index[magic]{Glowing Longsword}

This sword glows while unsheathed.  You have a bonus to attack and damage rolls made with this magic weapon.

\subsubsection{13}
\textbf{Description:}
\\
\\
\\
\textbf{Monsters:}
\\
\\
\\
\textbf{Traps:}
\\
\\
\\
\textbf{Treasure:}
\\
\\
\\

\subsubsection{14}
\begin{quotebox}
    This open area is a small worship alcove. On a raised platform
    along the western wall is a beautifully carved statue of a woman
    holding a small girl child in her lap. The woman is smiling down at
    inscription on the base of the statue reads “The secret treasure of
    one’s heart can be found in love.”
\end{quotebox}

A small opening beneath the inscription is the lock to open the
compartment in the base of the statue. One of the gold coin-like
octagons found in room \textbf{EL 6} will open it if inserted into the
opening.  Once opened, a scroll case will be found, and in it a fragment
of a verse written in silver ink on vellum parchment:

\begin{commentbox}{}
    I came, and what did my eyes behold? \\
    A maiden fair with hair of gold. \\
    Her face, aglow by which the sun is shamed. \\
    My steed, a dragon, her innocence did tame. \\
    Her heart, a gem with many facets.
\end{commentbox}

\subsubsection{15}
\begin{quotebox}
    In this small and once luxuriously decorated semi-circular room is a
    tiny 3’x3’ alcove in which stands a statue of a young girl with arm
    outstretched. The area seems peaceful.
\end{quotebox}

If the secret door is opened it will trigger a mechanism which will pour
down 200 cn worth of golden glitter upon the first person to step
through. This glitter will stick to all exposed skin, hair, leather and
cloth. It cannot be removed except by oil or animal fat. If players
attempt to wash it off with wine or water all they will succeed in doing
is rearranging it a little. The only way to avoid, this trap is to place
a weight of 600 cn on the pressure plate just inside the secret door.
This will set off the trap, and the glitter will stick to the floor,
instead. The glitter will glow in the dark, thus adding + 3 to the
chance of being surprised by any opponent who is in the line of sight.
    
\subsubsection{16}
\begin{quotebox}
    The first thing seen upon entering this room is a plaque that reads
    “All that glitters is not gold.” There is also a small fountain of
    water in one corner and both the north and south walls are covered
    by arrases. One arras has a scene of a young maiden with golden hair
    sitting on a silver throne. Upon her head rests a coronet of silver
    and rubies, and in her hand a scepter of silver topped by a very
    large blood red ruby. The arras show a warrior in blue and silver
    armor resting casually in a wooden arm chair decorated with
    carvings. His feet are propped up on a stool.
\end{quotebox}

\subsubsection{17}
\begin{quotebox}
    Four statues dominate the room, one in each cor- ner. Each one is of
    a young girl in a different pose. Between the two statues on the
    east wall is a kneeling bench, and on it rests an open book. Plush
    rugs that are still in fair condition cover the floor.
\end{quotebox}

The book is the diary of Lady Argenta. It simply tells of the fighter in
silver and blue armor coming to her home, winning her love and then
marrying her. It stops after the fourth day of their marriage. It does
mention “My Lady’s Heart” being somewhere in the living quarters of the
palace hidden in a teak wood jewel case.

\textbf{Monster:}
\\
\\
\\

\subsubsection{18}
\begin{quotebox}
    This area seems to have been in some kind of explosion or
    earthquake. Rubble covers the floor. Occasionally whimpers like
    those from a puppy can be heard. They frequently start only to stop
    a few seconds later.
\end{quotebox}

The whimpering is only the wind blowing through the rubble.

\textbf{Monster:}
\\
\\
\\

\subsubsection{19}
\textbf{Description:}
\\
\\
\\
\textbf{Monsters:}
\\
\\
\\
\textbf{Treasure:}
\\
\\
\\

%\begin{figure*}[hb]
%    \includegraphics[width=\textwidth]{img/wand.png}
%\end{figure*}
\includegraphics[width=\columnwidth]{img/wand.png}

\subsubsection{20}
\begin{quotebox}
    This very small chamber is more of a passageway than a room. It is
    very cramped and there are several sets of empty shelves on the
    walls.
\end{quotebox}

The pit trap in the floor of this storage passage will be activated by
the first person to step onto it, and triggered by the second one who
steps onto it. Once triggered, the floor will swing open and drop
whatever is on it into a 10’ deep pit. The cover then will swing back up
and lock shut. Anyone falling into the pit will take 1d6 points of
damage (DC 10 Dexterity saving throw for half damage).

If the first person who walks across the trap door is at the other side
before the second one tries, only the second one will fall in, otherwise
both will fall in.

After one round, small openings will appear in the walls, and
oil will pour out into the pit. The oil will continue to spill forth
until it lies 1” deep over the entire surface area of the floor. As
soon as this occurs, another wall opening will appear and an unlit torch
will fall onto the oil. (When the palace was occupied, the torch would
have been lit.)

Characters who are not trapped in the pit will be unable to open it by
any means other than using the release mechanisms hidden inside secret
compartments on the inside of either doorway. (Note that any character
covered in glitter from the secret door at room EL 15, who has fallen
into this pit, will discover that the glitter is coming off due to the
oil.)

\subsubsection{21}
\textbf{Description:}
\\
\\
\\
\textbf{Monsters:}
\\
\\
\\
\textbf{Treasure:}
\\
\\
\\

\subsubsection{22}
\begin{quotebox}
    This room is cluttered with many objects large and small.
\end{quotebox}

If any light source is brought into the room, eerie shadows begin to
dance wildly about. One shadow, lurking in the corner, appears to be
human or human-like.

The human-like form in the corner is actually a dressmakers’ dummy. The
room is filled with old bolts of cloth so rotten that merely brushing up
against them causes them to disintegrate into thousands of little
pieces. Also hidden in the room in a pin cushion ball is a small
delicate platinum needle (value 15 gp) brought to the Lady Argenta from
a far-away land. Metal needles are very rare, and platinum ones are
even rarer.

\textbf{Monsters:}
\\
\\
\\
\textbf{Treasure:}
\\
\\
\\

\subsubsection{23}
\begin{quotebox}
    Sand covers almost the entire floor of this once lavishly decorated
    room. Glints of silver may be seen in the sand near the center of
    the room.
\end{quotebox}

If the party searches the sand, the silvery glint will prove to be
strands of dancing bells on small delicate chains. There is a 10’ deep
pit near the center of the room, and players who go near it will have a
25\% (DC 10 Dexterity saving throw) chance of falling into the pit.
Since the trap door is sand covered there is only 1 chance in 8 (DC 15
Perception check) of being able to detect it before. Anyone falling in
will take 1d6 points of damage.

Also trapped in the pit is a \textbf{baric} who fell into it and is now
nearly dead from starvation. Due to the fact that it is half starved,
it will attack at a -2 on all “to hit” rolls.

Hidden in the sand in the north-western corner of the room is a small
sack of mixed coins (10 gp, 8 sp and 9 cp) and a jade ring with dragons
carved into it (value 250 gp). The ring is not magical.

\includegraphics[width=\columnwidth]{img/baric.png}

\begin{monsterbox}{Baric}
    \index[monsters]{Baric}
    \textit{Small beast, unaligned}\\
    \hline
    \basics[%
        armorclass = 12,
        hitpoints  = 7 (2d6),
        speed      = {30 ft.}
    ]
    \hline
    \stats[
        STR = \stat{7},
        DEX = \stat{15},
        CON = \stat{11},
        INT = \stat{2},
        WIS = \stat{10},
        CHA = \stat{4}
    ]
    \hline
    \details[
        senses = {darkvision 60 ft., passive Perception 10},
        languages = {---},
        challenge = {1/8 (25 XP)},
    ]
    \hline
    \\[1mm]
    \begin{monsteraction}[Keen Smell]
        The rat has advantage on Wisdom (Perception) checks that rely on
        smell.
    \end{monsteraction}

    \begin{monsteraction}[Pack Tactics]
        The baric has advantage on an attack roll against a creature if at
        least one of the rat's allies is within 5 feet of the creature
        and the ally isn't incapacitated.
    \end{monsteraction}
    \monstersection{Actions}
    \begin{monsteraction}[Bite]
        \textit{Melee Weapon Attack:} +4 to hit, reach 5 ft., one
        target.

        \textit{Hit:} 4 (1d4 + 2) piercing damage.
    \end{monsteraction}
\end{monsterbox}

\subsubsection{24}
\begin{quotebox}
    Upon entering this rectangular room, the first thing that will be
    noticed are the arrases hanging on all four walls and the many
    couches circled around a 5’ wide decorative wheel that is painted on
    the floor. Various pillows of many sizes (now musty and falling
    apart) are scattered randomly about the room. Crushed and punctured
    wine gob- lets are piled into one corner of the room. In another
    corner of this room stands a small lap harp that has no strings.
    Candle holders, a few of which are very decorative, sit on small
    tables, almost the size of stools, near each couch.
\end{quotebox}

The arrases are as rotten as most other materials so far encountered.
The couches are made of marble and have been cemented to the floor. The
seven decorative candle holders are made of silver (value 50 gp each).
The harp in the corner is an Ice Harp. If the players examine it
closely, they will discover that it is made of crystal, and though no
strings can be seen, when touched, sweet clear music will be heard. The
harp’s magical properties are such that any skilled harpist (a skill the
player characters may lack) playing it can calm and relax any beast
listening to it. The musical effects will begin 1 round after play has
begun. Saves vs. Spell are applicable. Its value is 600 gp.

\textbf{Monster:}
\\
\\
\\

\subsubsection{25}
\begin{quotebox}
    A statue of a small dragon readying for flight is leaning against
    the northeastern corner of this partly carved out room. A set of
    stairs going up is in the north wall. The whole room appears to have
    been cut from the living rock, instead of built from rocks brought
    in from mountain quarries. This area does not appear to be made from
    marble.
\end{quotebox}

There is a false doorway in the west wall placed there to trap
intruders. If the iron ring is grasped a poisoned needle will spring out
and pierce the hand of the grasper. A DC 10 Dexterity saving throw due
to the age of the venom must be made by the stricken character to avoid
death.

\textbf{Monster:}
\\
\\
\\
\textbf{Treasure:}
\\
\\
\\

\begin{figure*}[ht]
    \includegraphics[width=\textwidth]{img/travis.png}
\end{figure*}

\subsubsection{26}
\begin{quotebox}
    An overturned oaken table and three benches are all that remain in
    this small guard room. The floor is thickly covered in dust, and
    nothing seems to have disturbed it in a long time. There is a large
    sack in the southeast corner. Large blood stains are smeared on the
    floor beside it.
\end{quotebox}

Within the sack are three human skulls, a dagger, a dagger blade, and 11
sp. There are many vile-looking but harmless spiders living in the
skulls. Under the sack is a bloody finger joint. It appears to be fresh.

\textbf{Monster:}
\\
\\
\\

\subsubsection{27}
\begin{quotebox}
    This large rectangular room contains many implements of torture.
    An iron maiden hangs in one corner. Rusted, long-neglected
    branding irons lie scattered among the filthy, blood-stained straw.
    Assorted sizes and lengths of chain encircle several skeletons
    hanging limply against the walls. Small wooden cages hang from the
    ceiling. Caught in the door of one is a bit of what appears to be a
    tattered nightgown. Several mice peer out of holes and cracks in the
    grey stone walls.
\end{quotebox}

The room is empty when the party enters. After one round, a crazed old
man, \textbf{Travis} with a meat cleaver will come up the stairs from
the south and appear in the doorway. He will laugh insanely and then
attack the closest person to him. Travis, an old crazed warrior, will
scream at the players saying that he knew they could not resist his
treasure. No one could, he laughs, not even his companions. He knows
they have come to steal his great treasure, and so they all must die
just as others before them. He will attack until either he or the
characters are dead. He will neither surrender nor allow himself to be
captured.

No treasure can be found on him or in the room.

\begin{monsterbox}{Travis}
    \index[monsters]{Travis}
    \textit{Medium humanoid (human), chaotic evil}\\
    \hline
    \basics[%
        armorclass = 12,
        hitpoints  = 11 (2d8 + 2),
        speed      = {30 ft.}
    ]
    \hline
    \stats[
        STR = \stat{11},
        DEX = \stat{12},
        CON = \stat{12},
        INT = \stat{10},
        WIS = \stat{10},
        CHA = \stat{10}
    ]
    \hline
    \details[
        senses = {passive Perception 10},
        languages = {Common},
        challenge = {1/8 (25 XP)},
    ]
    \hline
    \\[1mm]
    \monstersection{Actions}
    \begin{monsteraction}[Meat Cleaver]
        \textit{Melee Weapon Attack:} +3 to hit, reach 5 ft., one
        target.

        \textit{Hit:} 4 (1d6 + 1) slashing damage.
    \end{monsteraction}
\end{monsterbox}

\subsubsection{28}
\begin{quotebox}
    A horrible smell like rotting carcasses can be detected beyond the
    door of this room.
\end{quotebox}

Once opened, mounds of rotten, decayed bodies of unlucky adventurers can
be seen covering almost every inch of the floor. The sight is gruesome
to behold, and characters with constitutions of less than 7 will not be
able to enter the room without becoming ill for 2-7 rounds (1d6 + 1)
from the smell and gory sight. The bodies have all been thoroughly
searched prior to the party finding them and there is nothing of value
to be found.

\subsubsection{29}
\begin{quotebox}
    A small pallet of fresh straw lies near the northwestern corner of
    this room. A wooden trencher, a pair of eating knives and a pewter
    wine goblet rest neatly on a table in the center of the room.
    Several old tapestries have been carelessly hung on the walls, and
    bits of fur and other types of floor covering form a makeshift
    rug. A burning lantern hangs over the table.
\end{quotebox}

This is the room where Travis lives. On the east wall, behind the
tapestry is the peephole he uses to spy out into the hallway. Hidden
underneath the pallet, protected by a loose stone in the floor, is a
small wooden case. This case contains Travis’ personal treasure;
\textbf{2 rubies (300 gp each), 1 large emerald (2000 gp), a gold
    wedding ring with the initials D and B carved in the shape of a
    heart on the inside (10 gp), and a gem-studded throwing dagger + 2
    (the plus only applies if the dagger is thrown at an opponent; its
value is 400 gp)}. The valuables he has removed from his victims are
hidden in room \textbf{EL 32D}.

\subtitlesection{+2 Throwing Knife}
{Weapon (any), uncommon}
\index[magic]{Throwing Knife}

You have a bonus to attack and damage rolls made with this knife when it
is thrown.

\subsubsection{30}
\begin{quotebox}
    Directly across from the northern door is a huge wooden table still
    in good condition. Behind it is a huge ornately carved wooden chair.
    On the table is a candle sconce, a feathered quill, a blank scroll,
    and a string of colored wooden beads.
\end{quotebox}

Travis kept this room in good shape. He used it to hold ‘court’ if he
decided to impress some of his victims. The paper scroll, pen, and
candle sconce are still in good condition and were obviously used by
Travis when he passed judgment on the accused. However, the beads will
be a mystery to the adventurers. These beads are message beads used by
the dead soldiers found in \textbf{EL 7}. The message depicted on the beads
must be determined by the DM.

\subsubsection{31}
\begin{quotebox}
    This rather large room has been swept clean. No dirt or dust can be
    found. The room is empty of all furnishings.
\end{quotebox}

Travis, it appears was a very clean man. All the rooms he claimed as his
territory were used for a specific reason or kept completely clean.

\subsubsection{32}
\begin{quotebox}
    This area is a group of jail cells. A few of them contain skeletons
    or corpses chained to the walls.
\end{quotebox}

All the cells are locked. The set of jailer’s keys mentioned earlier
(\textbf{EL 5}) will open all the cells. Note that this whole area would be an
excellent place to hide monsters and treasure.

\textbf{Cell D:} In this cell are 2 large marmoset monkeys who will
attack anyone who enters the room (except Travis). The marmosets stand
8’ high when erect; and have 12’ long tails that they use not only for
balance, but for attacking. The tails are tipped by sharp furry
spikes. They are protecting the treasure that Travis has collected
from his victims. All these items are locked in three large metal
chests. The first chest contains a large mixture of coins of several
different realms totaling 1000 cp, 400 sp, and 200 gp. The second chest
holds a variety of jewelry, mostly artificial or costume, worth 500 gp.
The last one is filled with swords, daggers, and helms. Only one of the
swords is magical, at a bonus of + 1. It is indistinguishable from the
rest of the swords unless a detect magic is cast on them.

\includegraphics[width=\columnwidth]{img/marmoset.png}

\begin{monsterbox}{Giant Marmoset}
    \index[monsters]{Giant Marmoset}
    \textit{Medium beast, unaligned}\\
    \hline
    \basics[%
        armorclass = 15,
        hitpoints  = 13 (3d8 + 3),
        speed      = {40~ft., climb 30~ft.}
    ]
    \hline
    \stats[
        STR = \stat{11},
        DEX = \stat{15},
        CON = \stat{13},
        INT = \stat{2},
        WIS = \stat{11},
        CHA = \stat{11}
    ]
    \hline
    \details[
        senses = {passive Perception 10},
        languages = {Common},
        challenge = {1 (200 XP)},
    ]
    \hline
    \\[1mm]
    \monstersection{Actions}
    \begin{monsteraction}[Tail Spike]
        \textit{Melee Weapon Attack:} +4 to hit, reach 5 ft., one
        target.

        \textit{Hit:} 2 (1d4) piercing damage.
    \end{monsteraction}

    \begin{monsteraction}[Bite]
        \textit{Melee Weapon Attack:} +4 to hit, reach 5 ft., one
        target.

        \textit{Hit:} 4 (1d8) piercing damage.
    \end{monsteraction}
\end{monsterbox}


\subtitlesection{+1 Shortsword}
{Weapon (any), uncommon}
\index[magic]{Shortsword}

You have a bonus to attack and damage rolls made with this magic weapon.

\subsubsection{33}
\begin{quotebox}
    This small cave is filled with ornate and delicately carved
    life-size statues of different men and women. Many candles and other
    burnt offerings lie before each of the statues. Marble benches form
    a circle in the center of the room.
\end{quotebox}

The statues represent unknown gods and goddesses. Hidden in one of the
statue’s arms is a \textbf{wand of light}. Players will have to
successfully search or detect for secret doors on the statues in order
to find the wand. The wand has only 3 charges left, and looks the same
as any other wand.

\subtitlesection{Wand of Light}
{Wand, rare}
\index[magic]{Wand of Light}

This wand can glow bright enough to light up a 100 sq. ft. for 1 hour
per charge.

\subsubsection{34}
\begin{quotebox}
    A single statue of embracing lovers dominates this cave. Dead vines
    and other plant life hang loosely to the rough walls. They were
    originally grown in clay pots, but have not received care for a long
    time. The floor is worn smooth.
\end{quotebox}

Nothing of value can be found in this area. The smoothness of the floor
is due to the hundreds of feet that have trekked in and out of this cave
over the previous centuries.

\subsubsection{35}
\begin{quotebox}
    This huge cave is filled with stalactites and stalagmites covered
    in a shimmering pink glow. The stalactites and stalagmites in many
    places have formed into one single column. It is very difficult to
    move through this area as the stalactites and stalagmites are very
    close together. In some places they almost form walls.
\end{quotebox}

If the players decide to investigate this area, they will discover that
the northern section of the cave is fairly free of stalagmites and
stalactites. In this empty space stands a statue of a beautiful woman
beckoning to any who approach. Anyone who manages to make their way
through the maze of stalactites and stalagmites may fall into a pit trap
(DC 15 Dexterity saving throw) that is placed at the base of the statue.
This 50’ deep pit is filled with stagnant water.

\subsubsection{36}
\begin{quotebox}
    Behind the bars of the entrance to this cave pitch black water can
    be seen, as well as a glint of gold from time to time. Hot winds
    seem to come from this barred area. The dampness on the walls is
    apparent from the droplets that fall to the floor. Moisture fills
    the air and clings to clothes, skin, and hair. The floor is slick
    from the warm water.
\end{quotebox}

All characters must succeed on a DC 15 Dexterity saving throw or  fall
due to the slipperiness of the floor during the first round in the room.
Thereafter they are safe, as they will grow accustomed to the wetness,
though not to the heat.  Once they have managed to lift the portcullis
(35 strength points are needed due to its rusted condition), it will
stay up.


\end{document}
