\documentclass[palace_of_the_silver_princess]{subfiles}

\begin{document}
\fontfamily{ppl}\selectfont
\clearpage

\twocolumn[{\section{Part 1: Introduction}}]
\markright{Part 1: Introduction}

A great many of the things found in the \textbf{Palace of the Silver
Princess} are there to add color and to give the DM ideas upon which to
expand. This module has been specially designed to give the beginning
DM, as well as the more experienced DM, a framework on which to build a
whole dungeon complex. This module can also be used as the basis for an
ongoing campaign, as it provides rumors, legends and other information
that give a campaign foundation and background. To expand the dungeon,
the DM need but open up the blocked passageways and add new and
challenging dungeon levels.  This should be done only after most of the
encounter areas have been explored.

Many of the rooms have spaces for monsters, treasure, and/or traps. Some
examples have been given of how to stock these rooms in other areas of
this module. By leaving some areas blank, the DM can use creativity to
add challenge to the module and make it fit into his or her world and
campaign. It also insures that even if some players read the module
before playing in it, they will not know exactly what is going to happen
in every room. Do not fill all of the rooms at once. Leave some empty to
be filled at a later time. This will help add color and suspense to the
adventure; a room visited earlier which proved to be empty and a
possible resting place might be occupied now by a monster that doesn’t
wish to share its room with adventurers. Many monsters and treasures can
be found in the DUNGEON \& DRAGONS® Basic Set Booklet. These are the ones
that should be used until the players have advanced past third level.
The new monsters and treasures found in this module should not be used
until the entire module has been explored, and the DM has drawn new maps
to expand the palace. These new monsters and treasures have been placed
in certain areas and play balance has been carefully considered in
placing them. If these monsters and/or treasures are moved elsewhere in
the module before the players discover them where they have originally
been placed, the module will become unbalanced and perhaps too
difficult, especially for first level adventurers. Once all the monsters
and treasures have been discovered, the DM may wish to place new
monsters and treasures elsewhere.

This module, like all DUNGEONS \& DRAGONS products, is a guideline to use
as a creative basis for your own campaign. It is designed to teach a new
DM how to design and run a D\&D adventure , while not being too difficult
for low level adventurers and new players. Good luck and enjoy.

\end{document}
