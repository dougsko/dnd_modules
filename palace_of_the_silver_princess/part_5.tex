\documentclass[palace_of_the_silver_princess]{subfiles}

\begin{document}
\fontfamily{ppl}\selectfont
\clearpage

\twocolumn[{\section{Part 5: Key To The Guard Tower Level}}]
\markright{Part 5: Key To The Guard Tower Level}

This entire level is guarded by the Protectors. No room will have a
monster, treasure or trap placed in it. The rooms are numbered and a
brief description of each is given. The DM may at a later time decide to
remove the Protectors and place monsters, treasure, and traps in this
area. This area is to be considered a resting place, a haven for lawful
characters. No chaotic or neutral character may enter here without being
challenged by a Protector. Lawful characters will be allowed to come
and go freely, and many never know that the Protectors are here.

\begin{enumerate}
    \item This is the top of the stairwell. A weapons rack is against the
        south wall. It holds two swords, a mace, and a dagger, all in fairly
        good condition.

    \item This room resembles an enclosed walkway.  Three windows line the
        west wall.  In the center of the room are two massive square columns.
        Four weapons racks are in this area, but all are empty.

    \item This strangely shaped room is divided into two sections, one
        facing south and the other west. Each section has one empty weapons rack
        and two windows. A large gong can be seen from two of the windows in an
        alcove.

    \item This room is sectioned off into two chambers. An empty weapons
        rack hangs against the north wall and the west wall. There are five
        windows on the southeastern wall.

    \item This large room has two empty weapons racks, a long one and a
        short one, and 4 windows on the east wall.

    \item This room has a large square column in the center, and two empty
        weapon racks, one on the east wall, the other on the south wall. There
        are 5 windows on the northwest wall.

    \item This is the gong that can be seen from room \textbf{TL 3}. If the
        players hit the gong, the sound can be heard for miles. It may also
        alert any wandering monster in the surrounding lands, thus causing a
        problem when the adventurers decide to head back to the city. The
        battlement encompasses the entire guard tower. All the windows have a
        glass-like substance secured over them to keep out cold, wind, and rain.
        The walkway around the battlements is clear.
\end{enumerate}

\begin{monsterbox}{Protector}
    \index[monsters]{Protector}
        \textit{Medium elemental, lawful good}\\
        \hline
        \basics[
                armorclass = {18},
                hitpoints = {74 (7d8 + 46},
                speed = {30~ft., fly 60~ft.}]
        \hline
        \stats[
                STR = \stat{15},
                DEX = \stat{11},
                CON = \stat{16},
                INT = \stat{6},
                WIS = \stat{11},
                CHA = \stat{7}]
        \hline
        \details[
            damageresistances = {acid, fire, lightning, thunder; bludgeoning, piercing, and slashing from nonmagical attacks},
            damageimmunities = {cold, necrotic, poison},
            conditionimmunities = {charmed, exhaustion, frightened, grappled, paralyzed, petrified, poisoned, prone, restrained},
                senses = {darkvision 60~ft., passive Perception 11},
                languages = {Communicates telepathically},
                challenge = {4 (1100 XP)}]
        \hline
        \monstersection{Actions}

        \begin{monsteraction}[Smash]
            \textit{Melee Weapon Attack:} +5 to hit, reach 5ft., one target.

            \textit{Hit:} 17 (4d6 + 3) bludgeoning damage.
        \end{monsteraction}
    \end{monsterbox}

    \subsection{Credits}
    \begin{description}
        \item[Design:] Jean Wells

        \item[Development:] Brian Pitzer, Jean Wells

        \item[Editing and Production:] Edward G. Sollers, Stephen D. Sullivan

        \item[Inspiration:] Harold Johnson and my father, Walt Wells

        \item[Much Thanks:] Frank Mentzer, Skip Williams, “Col.” Steve Austin
            Morely, John Laney, Robert Cole, Kevin Woods and
            a special thanks to Tony Earls

        \item[Artists:] Jeff Dee, David S. La Force, Erol Otus, Jim Roslof,
            Laura Roslof, Stephen D. Sullivan, Jean Wells,
            Bill Willingham

        \item[Playtesters:] Ken Reek, Jo La Force, Dave La Force, Judy Elvin,
            Skip Williams, Dave Conant, Shirly Egnoski, Ernie Gygax, John
            and Mary Eklund, Michael Luznicky, Blane Fuller, Jan Kratochvil,
            Mark Teloh, John Beckman, Bob Burgess, John Main, Gregory G.H.
            Rihn, Doug Jones, Bryan Wendorf, Tina Pacey, Rocky Bartlett and
            Helen Cook
    \end{description}


\end{document}
